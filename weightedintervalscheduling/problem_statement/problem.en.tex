\problemname{Weighted Interval Scheduling}

\illustration{.45}{img/sample-2-image.pdf}{%
A set of 8 weighted intervals corresponding to sample input~2.
The maximum total weight of nonoverlapping intervals is 12, achieved by the highlighted intervals.}

\noindent
Consider a set of $n$ weighted intervals $I_1,\ldots, I_n$, each given as an integer tuple $(s_i, f_i, w_i)$ with $s_i<f_i$, such that the $i$th interval starts at $s_i$, ends at $f_i$, and has weight $w_i$.
We want to determine the maximum total weight of nonoverlapping intervals.
More formally, we want to maximise $\sum_{i\in S} w_i$ over the subsets $S\subseteq \{1,\ldots,n\}$ such that for $i, j\in S$ with $i\neq j$ we have $f_j \leq s_i$ or $f_i\leq s_j$. 
Note that the intervals $[1,2]$ and $[2,3]$ are not considered to be overlapping.

\section*{Input}

The first line of input consists of an integer $n\in\{1,\ldots, 10^5\}$, the number of intervals.
Each of the following $n$ lines conists of 3 space-separated integers $s$, $f$, and $w$, indicating an interval starting at $s$, ending at $f$, and of weight $w$, with $0\leq s<f\leq 10^9$ and $1\leq w\leq 10^9$.


\section*{Output}

Output a single integer $t$, the maximum total weight of a subset of the nonoverlapping intervals.
You can assume  $1\leq t\leq 10^9$.
