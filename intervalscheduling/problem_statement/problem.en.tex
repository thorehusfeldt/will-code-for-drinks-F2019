\problemname{Interval Scheduling}

\illustration{.45}{img/sample-2-image.pdf}{%
A set of 8 intervals corresponding to sample input~2.
The maximum number of nonoverlapping intervals is 3, achieved by the highlighted intervals.}

\noindent
Consider a set of $n$ intervals $I_1,\ldots, I_n$, each given as an integer tuple $(s_i, f_i)$ with $s_i<f_i$, such that the $i$th interval starts at $s_i$, ends at $f_i$.
We want to determine the maximum number nonoverlapping intervals.
More formally, we find a largest subset $S\subseteq \{1,\ldots,n\}$ such that for $i, j\in S$ with $i\neq j$ we have $f_j \leq s_i$ or $f_i\leq s_j$. 
Note that the intervals $[1,2]$ and $[2,3]$ are not considered to be overlapping.

\section*{Input}

The first line of input consists of an integer $n\in\{1,\ldots, 10^5\}$, the number of intervals.
Each of the following $n$ lines conists of 2 space-separated integers $s$ and $f$, indicating an interval starting at $s$ and ending at $f$, with $0\leq s<f\leq 10^9$.


\section*{Output}

Output a single integer, the largest number of nonoverlapping intervals.
