\problemname{Cheaper Drink}

%\illustration{.3}{Yardley_Wood.jpg}{A wall calendar displaying the date 5 OCT. Image source: Yardley Wood Bus Garage - Open Day - signs - date - Oct 05.  Author:	Elliott Brown from Birmingham, United Kingdom. Wikimedia Commons.}

Instead of worrying about the current hyperinflation you decide to go down to the local bar and have a drink.

The prices at the bar are displayed using magnetic signs with numbers printed on them, with each magnet showing one more more digits.
For instance, the price of 1106 adjusted megacredits is displayed like this:

\medskip
\includegraphics[width = 2cm]{img/from.png}

While the bartender is busy serving the next customer, you have just enough time to rearrange the price of your favourite beverage to make it as cheap as possible.
But be quick about it!

Invidual magnets can be moved around in any order, or turned upside-down.
The numbers are shown in a script that makes it difficult for the bartender to distinguish 0, 1, and 8 from their upside-down counterpart.
Moreover, 6 and 9 look the same when one is turned upside-down.
The example price above could be turned into 116, almost ten times cheaper, by turning the first magnet:

\medskip
\includegraphics[width = 2cm]{img/to.png}

You have to use all the magnets, otherwise the bartender will immediately suspect fould play.

\section*{Input}

On the first line, the number $n$ of magnets, with $1\leq n\leq 100\,000$.
On the next $n$ lines, exactly one integer $p_i$ describing the number printed on the $i$th magnet, where $p_i>0$ consists of no more than 100 digits (from $0$, $1$, $\ldots$, $9$), for $i\in\{1,\ldots, n\}$.

\section*{Output}

A single line containing one integer, which is the smallest number than can be written using the magnets containing $p_1,\ldots,p_n$ in any order, possibly turned upside-down.
